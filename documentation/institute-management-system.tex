\documentclass[12pt,a4paper]{article}
\usepackage[a4paper,margin=0.75in]{geometry}
\usepackage[utf8]{inputenc}
\usepackage{newtxtext,newtxmath}
\usepackage{indentfirst}

\title{INSTITUTE MANAGEMENT SYSTEM}
\author{Chanchal Sadhukhan}
\date{February 2020}

\begin{document}

\maketitle

\section*{Objective}

Nowadays, every \textbf{Institute} have a \emph{large database} to maintain. Publishing \textbf{notice}, circulate class \textbf{notes} and keep track on the student and teacher \textbf{attendance}, preparing \textbf{question paper} for exam generating \textbf{results}, an institute generated a big database even in a month. Thus, In order to maintain such database and even to handle such kind of large activities, we need to have some automatically working system, which does not require much labor work and can be done easily with less error.

\section*{\Large{Existing System}}

In existing system, all the sections have their separate files that too is kept in documents. All the sections have to be maintained manually and \textbf{information} have to \textbf{store} by \textbf{hands}. In existing system, a bundle of files are maintained for every section which in results makes a cluster of files of only one section even. 
Existing system takes too much time to update the details but still results in less accurate work as compared to the given details. Existing system even have an increase the chances of losing files even. 

\section*{Proposed System}

In the proposed system, everything is designing and developing in such a way that it works automatically. Every part of the regular activities going on in an institution.  All the database of the attendance is created and is stored in the server automatically.
Notices can be directly uploaded for the appropriate audience. Students can get notes and can access from everywhere. The system can keep track the attendance and can identify students with poor attendance in real time.
\end{document}
